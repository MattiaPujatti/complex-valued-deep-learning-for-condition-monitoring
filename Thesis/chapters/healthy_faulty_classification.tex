\documentclass[../main.tex]{subfiles}	



\begin{document}
	
\chapter{Healthy-Faulty Signal Classification}

In the second part of this thesis, we will apply the complex-valued deep learning framework, developed in part 1, to a real world problem of condition monitoring in industrial applications.\\
Condition Monitoring is defined as \textit{the process of monitoring a parameter of condition in machinery (vibration, temperature, etc.), in order to identify a significant change which is indicative of a developing fault}.\\
In our technological and industrial society, the use and mastery of condition monitoring turned out to be of extreme importance, since it allows to be schedule maintenance, or other actions to be taken in order to prevent consequential damage and avoid its, often expensive, consequences. Furthermore, it can helps in increasing the lifespan of many engines, preventing faults that could develop in major failures.\\	
Condition Monitoring techniques are normally used on rotating equipment, auxiliary systems and other dynamic machinery (compressors, pumps, electric motors, etc.) while for static devices, usually, periodic inspections are sufficient.\\
In our analysis, we will propose a modification (or, better, an extension) to the existing approach to the strategy that is usually followed for rotating machines, i.e. \textbf{vibration analysis}.\\

What engineers do, is usually taking measurements on machine bearing casings with \textit{accelerometers} to measure the casing vibrations, sometimes provided also of electric transducers able to directly observe the rotating shafts and detect their radial (and axial) displacements. Data collected are then set of vibrational signals that can be analyzed and studied to detect clues of something bad that is happening. Vibration levels can then be compared with some historical baseline values (a "ground truth") derived after long periods of experiments, and in some cases compared with established standards such as load changes, to keep high the attention. Vibration limits can also be defined based on the machine design or components, knowing their fault frequencies of bearings.\\
Interpreting the vibration signal obtained is an elaborate procedure that requires specialized training and experience. With the technology progresses and increasingly complicated machines, simple comparisons and vibration limits are not anymore sufficient to guarantee the performances and accuracies needed by the companies. Luckily, also the techniques available have widely improved: thanks especially to the recent advances of machine and deep learning, signal analysis and classification is strongly simplified. Several state-of-the-art approaches have been developed, able to provide the vast majority of data analysis automatically, retrieving information instead of raw data.\\
In general one commonly employed technique is to examine the individual frequencies present in the signal. These frequencies, in fact, can correspond to certain mechanical components or specific malfunctions (e.g. unbalance or misalignment). So, by analyzing these frequencies and their harmonics, a specialist can in principle be able to identify the location or the type of a problem, sometimes even the cause. And this is possible weeks, even months, before the effective failure or damage of the apparatus, giving ample time to the technicians for replacing or repairing the machine. \\
But the interesting part in this approach is that we have a mathematical instrument that allow us to move from the time to the frequency domain, with all the consequent advantages: the \textbf{Fourier Transform}. We will see, in the next section, that this operator not only permits the analysis of the frequency spectrum of the signal, but also represents the connection among the real-valued measurements of physical quantities and the complex-valued domain in which we can effectively put to test the framework we have developed.\\
It is however better to specify that frequency analysis tends to be most useful on machines that employ rolling element bearings and whose main failure modes tend to be the degradation of those bearings, which typically exhibit an increase in characteristic frequencies associated with its geometries and constructions.\\
Tons of different "manual" analysis are possible for those kind of signals in the frequency domain, a study of its phase spectrum for example, but in this thesis we are not really interested in them since we will rely on modern deep learning approaches, providing also efficient alternatives to the actual state-of-the-art techniques.


\section{State-of-the-Art}

In chapter \ref{}, we were complaining about the fact that literature does not provide almost any kind of complex-valued datasets to effectively test the models we have developed. There, we were forced to "manually construct" our data, with simple distributions of points \ref{}, or modifications of existing ones \ref{}. Relying on them could have been fine for that preliminary analysis, mainly focused on the convergence and the stability of the training process, rather then on the final performances. However, we believe that, in order to prove the efficacy of our method, we should stress our complex-valued architectures on more concrete and reliable datasets.\\
But is there the possibility of measuring in some way physical quantities that are inherently complex-valued? From this point of view, the answer is yes, and we just need to think about magnetic momenta. The PHD thesis we have based our work on \ref{}, is constructed exactly on these kind of data: Magnetic Resonance Images contains, in fact, much information in their phase, that with a complex neural network the author managed to outperform many existing approaches that used to discard such information.\\
But we should not limit our horizons when we have a powerful instrument like the Fourier transform $\mathcal{F}$, that allow us to

Several papers have been published, suggesting complex-valued models to study electric, seismic \ref{} and vibrational signals in the complex domain. So, in principle, for any signal registered in the time domain, we can recover, and maybe exploit, the phase information thanks exactly to the Fourier transform.\\
Studying the problem in the frequency domain is a quite common approach to signal processing, since many operations in the time domain have a complex counterpart that sometimes turns out to be easier to perform (eg. differentiation and convolution). Furthermore, also from a purely theoretical point of view, several concepts are easier to derive and understand with complex notations. 	



\section{Mendelay Data}


\section{Bonfiglioli}

\subsection{Simulation Environment}

\subsection{Datasets}


	
\end{document}